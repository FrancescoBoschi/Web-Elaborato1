\documentclass[a4paper,12pt]{report}
\usepackage{xcolor}
\usepackage{graphicx}
\usepackage{enumerate}
\usepackage[utf8]{inputenc}
\usepackage{hyperref}
\usepackage[italian]{babel}
\usepackage{listings}
\usepackage{color}
\lstset{
  basicstyle=\ttfamily,
  columns=fullflexible,
  frame=single,
  breaklines=true,
  postbreak=\mbox{\textcolor{red}{$\hookrightarrow$}\space},
  basicstyle={\small\ttfamily},
  numberstyle=\tiny\color{gray},
  keywordstyle=\color{blue},
  commentstyle=\color{dkgreen},
  stringstyle=\color{mauve},
  morekeywords={xmlns,version,type, rdf, dc, dcat}
}
\definecolor{dkgreen}{rgb}{0,0.6,0}
\definecolor{gray}{rgb}{0.5,0.5,0.5}
\definecolor{mauve}{rgb}{0.58,0,0.82}
\graphicspath{ {./images/} }
\hypersetup{
    colorlinks = false,
    linkbordercolor = {white},
    citebordercolor = {white}
}
\title{%
  GlobalStat \\
  \large Web Semantico - Primo elaborato \\}
\author{Boschi Francesco - 0000939879}
\date{\today}

\begin{document}

\maketitle

\tableofcontents


\chapter{Introduzione}
 Nel 2010, l'Istituto Universitario Europeo \cite{istituto_universitario_europeo} ha dato il via al progetto Global Governance Programme \cite{global_governance_programme}, il quale ha come obbiettivo principale la nascita di una comunità di professori, studenti e ricercatori di alto livello che contribuiscano e siano di supporto alle future generazioni in ambito politico e non solo.
 
 GlobalStat \cite{global_stats_website} nasce grazie ad una collaborazione con l'istituto Fundação Francisco Manuel dos Santos \cite{istituto_portoghese_francisco_manuel_dos_santos}, ovvero il primo finanziatore, come sottosezione del Global Governance Programme.\newline
 Dal momento della sua nascita ad oggi, GlobalStat ha stabilito e intensificato nuovi programmi di partnership, in quanto riconosciuto come risorsa fondamentale in ambito pubblico e accademico.
 Dal 2016 è infatti partner con l'Organizzazione per la Cooperazione e lo Sviluppo Economico (OCED) \cite{OCED} con l'obbiettivo di produrre nuove tecniche di gestione e visualizzazione dei dati necessari ad analizzare e favorire lo sviluppo economico.\newline
 Anche lo stesso Servizio Ricerca del Parlamento europeo \cite{servizio_di_ricerca_parlamento_europeo} ha deciso di allacciare rapporti con GlobalStat per ottenere statistiche ed infografiche utili allo staff del Parlamento e a tutti i paesi dell'Unione Europea uniti ai loro partner globali.
 
 Attualmente GlobalStat coopera con oltre 80 istituzioni internazionali, come Eurostat, l'Organizzazione per il Cibo e l'Agricoltura e l'Organizzazione Internazionale del Lavoro, con le quali vi è un continuo scambio di dati bidirezionale; le organizzazioni, oltre a sfrtuttare gli strumenti messi a disposizione da GlobalStats, forniscono dati e statistiche utili a mantenere l'ecosistema aggiornato.
\chapter{Obbiettivi}
Come afferma Gaby Umbach, direttore della fondazione GlobalStats \cite{global_stats_website}, la statistica ricopre un ruolo fondamentale in numerosi ambiti della nostra vita, da quello politico a quello sociale.\newline
Tale importanza si traduce in un aumento della domanda di dati statistici, che siano affidabili e pubblicamente disponibili, fattore spesso assente e fondamentale, specialmente nell'era della globalizzazione dove la velocità alla quale i dati vengono prodotti rende impossibile la loro gestione e analisi in maniera manuale.\newline
Le fitte interconnessioni derivanti dalla globalizzazione portano ad avere forte impatto in ambito sociale, personale, culturale, politico, economico e ambientale, rendendo i dati che GlobalStat raccoglie vitali per poter gestire al meglio le risorse e le possibilità che ciascun paese individuo possiede.

GlobalStat si pone quindi come obbiettivo quello di soddisfare questa esigenza di informazioni trasparenti e disponibili pubblicamente, così che possano essere reperite e visualizzate in maniera semplice e intuitiva da chiunque ne abbia necessità.\newline
Considerando l'obbiettivo alla base del progetto e il carattere multidimensionale della globalizzazione, GlobalStat presenta dati su una vasta  gamma di argomenti, ad esempio:

\begin{itemize}
  \item Dati demografici
  \item Dati relativi all'economia
  \item Dati relativi all'energia e alle risorse naturali
  \item Dati relativi all'ambiente e all'inquinamento
  \item Dati relativi alla produzione nei vari settori
  \item Dati relativi alla libertà, ai conflitti e ai pericoli
  \item Dati governativi
  \item Dati relativi alla salute e alle condizioni di vita
  \item Dati relativi a fattori etici e morali, come crimini, giustizia, diritti, educazione, uguaglianza e condizioni lavorative
  \item Dati relativi alle migrazioni
  \item Dati relativi allo sviluppo tecnologico
  \item Dati relativi a trend e mode
\end{itemize}

Emerge quindi come GlobalStat non miri solo e migliorare il lato strettamente economico di un paese, ma anche a informare sul modo in cui le persone vivono, le libertà di cui godono e le restrizioni che devono affrontare quotidianamente.

\section{Presentazione del sito WEB}

Il sito web si presenta con un'interfaccia tanto semplice quanto chiara e intuitiva, che vede sulla parte sinistra il menù, suddiviso in categorie e sottocategorie con tutti gli indicatori disponibili, mentre in quella destra alcune voci per ottenere informazioni sul progetto.

Una volta selezionato l'indicatore, di default vengono mostrate le statistiche in modalità mappa, dalla quale è possibile selezionare lo stato di interesse per visualizzare le informazioni specifiche, oltre che applicare i filtri messi a disposizione.\newline
La visuale ad istogramma rappresenta gli stessi dati sotto forma differente.

La modalità di visualizzazione "Trend" permette di avere una visione chiara dell'andamento dell'indicatore con il passare degli anni, a livello globale o nel paese selezionato tramite gli appositi filtri.

L'ultima modalità di visualizzazione è certamente quella più riutilizzabile per sviluppare progetti esterni a GlobalStat, in quanto permette non solo di consultare, ma anche di scaricare i dati relativi all'indicatore selezionato dettagliati per ciascun paese e anno.\newline
Il download viene fatto in formato xlsx, quindi sotto forma di foglio di calcolo Excel. Proprio per questo motivo, in questa sezione non sono sfruttate particolari tecnologie relative al web semantico.

Ciascuna sezione analizzata fino ad ora presenta inoltre un'icona alla cui pressione un popup permette all'utente di ottenere informazioni dettagliate sul significato dell'indicatore, note aggiuntive e soprattutto la fonte dalla quale è stato reperito, che spesso, come analizzato nel capitolo 4, fa ampio uso delle tecnologie del web semantico.

\begin{figure}[h]
  \includegraphics[width=\textwidth,height=\textheight,keepaspectratio]{globalStat.PNG} 
  \caption{Homepage del sito GlobalStat.}
\end{figure}

\begin{figure}[h]
  \includegraphics[width=\textwidth,height=\textheight,keepaspectratio]{mappa.PNG} 
  \caption{Mappa del sito GlobalStat.}
\end{figure}

\begin{figure}[h]
  \includegraphics[width=\textwidth,height=\textheight,keepaspectratio]{trend.PNG} 
  \caption{Trend del sito GlobalStat.}
\end{figure}

\begin{figure}[h]
  \includegraphics[width=\textwidth,height=\textheight,keepaspectratio]{data_table.PNG} 
  \caption{Tabella del sito GlobalStat.}
\end{figure}

\begin{figure}[h]
  \includegraphics[width=\textwidth,height=\textheight,keepaspectratio]{info_global.PNG} 
  \caption{Info del sito GlobalStat.}
\end{figure}

\chapter{Metodologie e sorgenti dati}
I dati raccolti sono sensibili al carattere di vasta portata del progetto; per questo motivo è stato necessario adottare apposite metodologie atte a produrre statistiche effettivamente utili e soprattutto il più possibile affidabili.\newline
GlobalStats raccoglie dati sui 193 stati nazionali sovrani riconosciuti membri delle Nazioni Unite (ONU) \cite{ONU} e, sulla base di questi, fornisce una panoramica sulle prestazioni dei singoli stati, dei continenti, di undici comunità di cooperazione e integrazione regionale e di organizzazioni internazionali.\newline
Questa forte cooperazione con le comunità è il pilastro principale dell'aggregazione dei dati effettuata da GlobalStat, in quanto tramite esse gli stati si impegnano in una collaborazione regionale per sostenersi a vicenda e migliorare sviluppo e condizioni di vita.

Sebbene tendenzialmente i confini tra stati rimangano invariati nel tempo, per tenere traccia di eventuali separazioni o formazione di nuovi territori, GlobalStat calcola in maniera dinamica i dati aggregati adattandosi al formato con cui vengono resi disponibili. Esempio lampante in cui questa metodologia ha permesso di avere dati coerenti nonostante le vicende politiche è la dissoluzione della Cecoslovacchia avvenuta nel 1993.\newline
Anche in fase di visualizzazione, quindi, i dati rispecchiano di anno in anno la composizione degli stati facenti parte dell'ONU, rendendo più complesso il calcolo di medie e la possibilità di effettuare comparazioni nel tempo.

Seguendo la prassi di statistica internazionale, GlobalStat calcola e rende disponibile dati aggregati se sono soddisfatte due condizioni:

\begin{enumerate}
  \item L'indicatore in questione è disponibile per più della metà dei membri del gruppo
  \item La popolazione totale dei paesi per i quali sono disponibili i dati rappresenta almeno 2/3 della popolazione totale del gruppo
\end{enumerate}
Indipendentemente da tali condizioni, seguendo un approccio prudenziale, i dati per Asia ed Europa vengono calcolati a partire dal 1992, anno dello scioglimento dell'URSS.\newline
Questa scelta è dovuta al fatto che i confini geografici dell'URSS erano condivisi tra due continenti, per i quali non è possibile dedurre il contributo dell'Unione in maniera isolata.

Se le condizioni 1 e 2 sono soddisfatte, GlobalStat applica metodologie di calcolo degli aggregati differenti dipendentemente dalla tipologia di dati:

\begin{itemize}
  \item Gli aggregati dei dati dei singoli paesi, se espressi come valori assoluti, sono calcolati semplicemente tramite la somma ignorando i valori mancanti
  \item  Gli aggregati dei dati dei singoli paesi, se espressi come valori di rapporto (tassi/proporzioni/percentuali), sono calcolati come medie ponderate sulla base del conteggio della popolazione sul totale, ignorando i valori mancanti
\end{itemize}
GlobalStat si affida a ben 113 fonti differenti di dati dipendentemente dagli indicatori necessari e dai paesi di riferimento.\newline
Un utente che consulta GlobalStat dovrebbe essere consapevole di questa peculiarità multi-fonte e delle metodologie sopra elencate, ed essere quindi cauto nel confrontare dati e indicatori differenti, poiché potrebbero potenzialmente variare e presentare incoerenze dovute all'intervallo temporale, alle pratiche di raccolta e dalla loro specificità.


\chapter{Architettura e tecnologie utilizzate}
GlobalStat, come analizzato in precedenza, funge principalmente da aggregatore di numerosissimi dati provenienti da sorgenti differenti, le quali possono essere a loro volta aggregatori piuttosto che semplici banche dati.\newline
Nel primo caso GlobalStat produce le proprie statistiche a partire dai file riassuntivi tabulari, resi disponibili per esempio in formato .xlsx, facendo uso ridotto delle tecnologie del web semantico in maniera diretta in quanto precedentemente sfruttate dall'aggregatore di riferimento.\newline
In tutti gli altri casi, invece, nei quali le stime devono essere effettuate fondendo dati provenienti da fonti differenti, è fondamentale l'utilizzo di tecnologie per disambiguare il significato delle risorse e interconnetterle all'interno del web.

\section{RDF, RDFS}
RDF è la tecnologia alla base della sezione open data del progetto, fondamentale per descrivere e mettere in relazione tra loro le risorse all'interno del web tramite l'utilizzo di triple.\newline
Tendenzialmente, in aggiunta ad RDF, viene utilizzato RDFS, che fornisce un ulteriore livello di conoscenza connettendo i concetti forniti tramite RDF alle primitive di corrispondenza (Class, Subclass, Property ecc.), permettendo così di fare inferenza e interrogare il dominio di appartenenza. 
Le seguenti analisi prendono come riferimento una risorsa messa a disposizione da World Bank \cite{data_catalog}, un'istituzione finanziaria internazionale che fornice dati a paesi di tutto il mondo ed è quindi una delle principali sorgenti sfruttate da GlobalStat.\newline

\begin{lstlisting}
<?xml version="1.0" encoding="UTF-8"?>
  <rdf:RDF xmlns:rdf="http://www.w3.org/1999/02/22-rdf-syntax-ns#"
    xmlns:content="http://purl.org/rss/1.0/modules/content/"
    xmlns:dc="http://purl.org/dc/terms/"
    xmlns:dcat="http://www.w3.org/ns/dcat#"
    xmlns:sioc="http://rdfs.org/sioc/ns#">

    <rdf:Description rdf:about="https://datacatalog.worldbank.org/dataset/global-economic-prospects">
      <rdf:type rdf:resource="http://rdfs.org/sioc/ns#Item"/>
      <rdf:type rdf:resource="http://xmlns.com/foaf/0.1/Document"/>
      <content:encoded>&lt;p&gt;The Global Economic Prospects</content:encoded>
      <dc:creator></dc:creator>
      <dc:Frequency>3</dc:Frequency>
      <dcat:granularity></dcat:granularity>
      <dc:license>notspecified</dc:license>
      <dc:accessRights></dc:accessRights>
      <dcat:Distribution rdf:resource="https://datacatalog.wb.org/dataset/global-economic/resource/ID"/>
      <dcat:Distribution rdf:resource="https://datacatalog.wb.org/dataset/global-economic/resource/ID"/>
      <dcat:Distribution rdf:resource="https://datacatalog.wb.org/dataset/global-economic/resource/ID"/>
      <dcat:Distribution rdf:resource="https://datacatalog.wb.org/dataset/global-economic/resource/ID"/>
      <dcat:Distribution rdf:resource="https://datacatalog.wb.org/dataset/global-economic/resource/ID"/>
      <dcat:Distribution rdf:resource="https://datacatalog.wb.org/dataset/global-economic/resource/ID"/>
      <dcat:Distribution rdf:resource="https://datacatalog.wb.org/dataset/global-economic/resource/ID"/>
      <dcat:Distribution rdf:resource="https://datacatalog.wb.org/dataset/global-economic/resource/ID"/>
      <dcat:theme rdf:resource="https://datacatalog.wb.org/taxonomy_term/2031"/>
      <dc:temporal>2018 - 2022</dc:temporal>
      <dc:title>Global Economic Prospects</dc:title>
      <dc:date rdf:datatype="http://www.w3.org/2001/XMLSchema#dateTime">2017-01-31T00:28:40-05:00</dc:date>
      <dc:created rdf:datatype="http://www.w3.org/2001/XMLSchema#dateTime">2017-01-31T00:28:40-05:00</dc:created>
      <dc:modified rdf:datatype="http://www.w3.org/2001/XMLSchema#dateTime">2021-02-09T00:14:01-05:00</dc:modified>
      <sioc:num_replies rdf:datatype="http://www.w3.org/2001/XMLSchema#integer">0</sioc:num_replies>
    </rdf:Description>
  </rdf:RDF>
\end{lstlisting}

Il file XML in questione fa riferimento ad una specifica risorsa disponibile nel catalogo di WordBank, raggiungibile tramite l'apposito URI presente in \textbf{rdf:about}.\newline
A partire dal tag \textbf{rdf:type} sono elencate tutte le proprietà della risorsa. Queste ultime rappresentano delle caratteristiche che, per essere riconosciute in maniera standard e inconfutabile, sono connesse ad un vocabolario che ne determina l'ontologia. Nel caso specifico vengono sfruttati tre dizionari differenti, ciascuno relativo ad un ambito ed in grado di disambiguare al meglio il significato della proprietà.

Il sistema di metadati Dublin Core \cite{DC} (identificabile dalla proprietà \textbf{dc:}) definisce i quindici elementi core per rappresentare le risorse, vale a dire "Creator", "Date", "Description", "Format" ecc.\newline
Il sistema Data Catalog Vocabulary \cite{DCAT} (identificato da \textbf{dcat:}) è invece utilizzato per descrivere i dataset e i servizi all'interno di un catalogo utilizzando un modello e un vocabolario standard, così da facilitare l'aggregazione (fondamentale all'interno di GlobalStat) e la consultazione.\newline
L'ultima specifica di ontologie utilizzata è la Semantically-Interliked Online Communities \cite{SIOC} (\textbf{sioc:}), che fornisce i concetti e le proprietà principali per descrivere informazioni relative a comunità online.

Questo esempio è solo un caso specifico selezionato tra le centinaia di sorgenti alle quali GlobalStat fa riferimento e alla miriade di risorse in esse contenute.\newline
XML non è l'unico formato disponibile ma ciascuna sorgente sfrutta quelli che ritiene più consoni. WordBank stesso permette di reperire i dati anche sotto forma di JSON, venendo così il più possibile incontro alle esigenze e alle preferenze dell'utilizzatore.\newline
In maniera analoga, anche la scelta dei vocabolari per le ontologie varia e viene fatta dipendentemente dagli elementi che devono essere rappresentati all'interno della risorsa e al grado di specificità richiesto dall'ente.

\section{OWL}
Sia DCAT che SIOC sono definiti sfruttando \textbf{OWL}, attualmente disponibile alle versione 2.\newline
OWL 2 Web Ontology Language \cite{OWL}, abbreviato OWL 2, è un linguaggio per le ontologie utilizzato nel web semantico. Le ontologie forniscono classi, proprietà e valori che sono memorizzati sotto forma di documenti del web semantico.\newline
Le ontologie fornite da OWL possono essere combinate tra di loro, oltre che essere utilizzate per arricchire informazioni contenute in documenti RDF.\newline
OWL è un linguaggio computazionale basato sulla logica: la conoscenza espressa tramite esso può essere dedotta da software i quali potranno inoltre verificare la consistenza e fare inferenza di nuove informazioni.

Per quanto concerne DC, invece, essendo uno dei primi progetti nati in ambito ontologie, non è definito mediante OWL.\newline
La differenza nel linguaggio rende più difficoltosa l'integrazione di DC con i dizionari più recenti; inoltre, con il passare degli anni sono emerse numerose limitazioni come l'utilizzo di  "LastName" e "FirstName" per identificare il creatore di una risorsa.\newline
Possono infatti essere presenti numerosi casi di omonimia all'interno di un singolo database, oltre alla difficoltà a disambiguare il nome dal cognome.\newline
Per questi motivi, negli ultimi anni stanno nascendo numerosi progetti atti a costruire versioni di DC basate su OWL, facilitandone così l'integrazione con ontologie definite ad hoc e disambiguando ulteriormente i termini \cite{DC_with_OWL}.

Nonostante le numerose limitazioni analizzate, vista l'ampia diffusione dovuta alla sua longevità, Dublin Core è tutt'ora uno dei vocabolari più diffusi per identificare i creatori di risorse.

\chapter{Commenti finali}
GlobalStat è sicuramente un progetto ambizioso, viste le difficoltà e i problemi che costantemente deve affrontare per portare statistiche affidabili, oltre che ammirevole in quanto si pone come obbiettivo principale quello di migliorare nel complesso la vita delle persone.\newline
Quest'ultima, tra tutte le motivazioni, è stata quella che mi ha maggiormente spinto ad approfondire questo caso d'uso.\newline
Attualmente GlobalStat funge principalmente da aggregatore: grazie alla sua folta rete di collaborazioni ottiene dati su scala globale, spesso a loro volta già precedentemente processati, e produce tramite essi statistiche che mette gratuitamente a disposizione degli utenti, principalmente in modalità visuale.

Non vi è quindi la possibilità di reperire i dati direttamente in maniera programmata, ma il sito stesso effettua un reindirizzamento alla sorgente originale, la quale adotterà poi delle specifiche politiche e tecnologie per l'accesso ai dati.\newline
Sebbene per gli enti governativi e le organizzazioni internazionali i dati siano già accessibili in maniera automatica tramite un agente software, la speranza è che in futuro tale possibilità venga estesa a chiunque, permettendo così una diffusione di queste importanti informazioni su scala ancora più ampia.

Inoltre, come tendenzialmente accade in statistica, l'importanza e il valore crescono insieme alla quantità di dati disponibili: proprio per questo, negli anni a venire, i vantaggi che GlobalStat sarà in grado offrire emergeranno maggiormente e sarà quindi fondamentale che tali dati siano accessibili in maniera semplice e automatizzata.


\bibliographystyle{unsrt}
\bibliography{bibliography}
\end{document}
