\documentclass[a4paper,12pt]{report}
\usepackage{xcolor}
\usepackage{graphicx}
\usepackage{enumerate}
\usepackage[utf8]{inputenc}
\usepackage{hyperref}
\usepackage[italian]{babel}
\graphicspath{ {./images/} }
\hypersetup{
    colorlinks = false,
    linkbordercolor = {white},
    citebordercolor = {white}
}
\title{%
  GlobalStat \\
  \large Web Semantico - Primo elaborato \\}
\author{Boschi Francesco - 0000939879}
\date{\today}

\begin{document}

\maketitle

\tableofcontents


\chapter{Introduzione}
 Nel 2010 l'Istituto Universitario Europeo \cite{istituto_universitario_europeo} ha dato il via al progetto Global Governance Programme \cite{global_governance_programme}, il quale ha come obbiettivo principale la nascita di una comunità di professori, studenti e ricercatori di alto livello che contribuiscano e siano di supporto alle future generazioni in ambito politico e non solo.
 
 GlobalStat \cite{global_stats_website} nasce, grazie ad una prima collaborazione con l'istituto Fundação Francisco Manuel dos Santos \cite{istituto_portoghese_francisco_manuel_dos_santos} che è stato di fatto il primo finanziatore, come sottosezione del Global Governance Programme.\newline
 Dal momento della sua nascita ad oggi, GlobalStat ha stabilito e intensificato nuovi programmi di partnership, in quanto riconosciuto come risorsa fondamentale in ambito accademico e pubblico.
 Dal 2016 GlobalStat è infatti partner con l'Organizzazione per la cooperazione e lo sviluppo economico (OCED) \cite{OCED} con l'obbiettivo di produrre nuove tecniche di gestione e visualizzazione dei dati.\newline
 Anche lo stesso Servizio ricerca del Parlamento europeo \cite{servizio_di_ricerca_parlamento_europeo} ha deciso di allacciare rapporti con GlobalStat per ottenere statistiche ed inforgrafiche utili  allo staff del Parlamento così come facilitare l'accesso a tali dati a tutti i paesi dell'Unione Europea e ai loro partner globali.
 
 Attualmente GlobalStat coopera con oltre 80 istituzioni internazionali, come Eurostat, l'Organizzazione per il Cibo e l'Agricoltura e l'Organizzazione Internazionale del Lavoro, con le quali vi è un continuo scambio di dati bidirezionale; le organizzazioni oltre a sfrtuttare gli strumenti messi a disposizione da GlobalStats, forniscono in continuazione statistiche utili a mantenere l'ecosistema aggiornato.
\chapter{Obbiettivi}
Come afferma Gaby Umbach, direttore della fondazione GlobalStats \cite{global_stats_website}, la statistica ricopre un ruolo fondamentale in numerosi ambiti della nostra vita, da quello politico a quello sociale.\newline
Tale importanza si traduce in un aumento della domanda di tali risorse, che siano affidabili e pubblicamente disponibili, fattore spesso assente e fondamentale, specialmente nell'era della globalizzazione nella quale la velocità alla quale i dati vengono prodotti rende impossibile la loro gestione e analisi in maniera manuale.\newline
Le fitte interconnessioni derivanti dalla globalizzazione portano ad avere forte impatto in ambito sociale, personale, culturale, politico, economico e ambientale, rendendo i dati che GlobalStat raccoglie vitali per poter gestire al meglio le risorse e le possibilità che ciascun paese e ciascun individuo possiede.

GlobalStat si pone quindi come obbiettivo di soddisfare questa esigenza di informazioni trasparenti e disponibili pubblicamente, così che possano essere reperite e visualizzate in maniera semplice e intuitiva da chiunque ne abbia necessità.\newline
Considerando l'obbiettivo alla base del progetto e del carattere multidimensionale della globalizzazione, GlobalStat presenta dati su una vasta  gamma di argomenti, ad esempio:

\begin{itemize}
  \item Dati demografici;
  \item Dati relativi all'economia;
  \item Dati relativi all'energia e alle risorse naturali;
  \item Dati relativi all'ambiente e all'inquinamento;
  \item Dati relativi alla produzione nei vari settori;
  \item Dati relativi alla libertà, ai conflitti e ai pericoli;
  \item Dati governativi;
  \item Dati relativi alla salute e alle condizioni di vita;
  \item Dati relativi a fattori etici e morali, come crimini, giustizie, diritti, educazione, uguaglianza e condizioni lavorative;
  \item Dati relativi alle migrazioni;
  \item Dati relativi allo sviluppo tecnologico;
  \item Dati relativi a trend e mode.
\end{itemize}

Emerge quindi come GlobalStat non miri solo e migliorare il lato strettamente economico di un paese, ma anche a informare sul modo in cui gli uomini vivono, le libertà di cui godono e i limiti che devono affrontare.
\chapter{Metodologie e sorgenti dati}
I dati raccolti sono sensibili al carattere di vasta portata del progetto; per questo motivo è stato necessario provvedere apposite metodologie atte a produrre statistiche effettivamente utili e soprattutto il più possibile affidabili.\newline
GlobalStats fornisce dati sui 193 stati nazionali sovrani riconosciuti membri delle Nazioni Unite (ONU) \cite{ONU} e, sulla base di questi, fornisce una panoramica sulle prestazioni dei singoli stati, dei continenti, di undici comunità di cooperazione e integrazione regionale e di organizzazioni internazionali.\newline
Questa forte cooperazione con le comunità è il pilastro principale dell'aggregazione dei dati effettuata da GlobalStat, in quanto tramite esse gli stati si impegnano in una collaborazione regionale per sostenersi a vicenda e migliorare lo sviluppo e le condizioni di vita.

Sebbene tendenzialmente i confini tra stati siano statici nel tempo, per tenere traccia di eventuali separazioni o formazione di nuovi stati, GlobalStat calcola in maniera dinamica i dati aggregati nel caso in cui le fonti originali li offrano formato, esempio lampante la dissoluzione della Cecoslovacchia fino al 1993.\newline
Anche in fase di visualizzazione quindi, i dati, rispecchieranno di anno in anno la composizione degli stati facenti parte dell'ONU, rendendo più complesso il calcolo di medie e la possibilità di effettuare comparazioni nel tempo.

Seguendo la prassi di statistica internazionale, GlobalStat calcola e rende disponibile dati aggregati se sono soddisfatte due condizioni:

\begin{enumerate}
  \item L'indicatore in questione è disponibile per più della metà dei membri del gruppo;
  \item Se la prima condizione è soddisfatta, il valore aggregato viene calcolato solo se la popolazione totale dei paesi per i quali sono disponibili i dati rappresenta almeno 2/3 della popolazione totale del gruppo.
\end{enumerate}

\chapter{Architettura e tecnologie utilizzate}
\chapter{Risultati}
\chapter{Commenti finali}

\bibliographystyle{unsrt}
\bibliography{bibliography}
\end{document}
